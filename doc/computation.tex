% Created 2012-05-14 Mon 19:02
\documentclass[11pt]{article}
\usepackage[utf8]{inputenc}
\usepackage[T1]{fontenc}
\usepackage{fixltx2e}
\usepackage{graphicx}
\usepackage{longtable}
\usepackage{float}
\usepackage{wrapfig}
\usepackage{soul}
\usepackage{textcomp}
\usepackage{marvosym}
\usepackage{wasysym}
\usepackage{latexsym}
\usepackage{amssymb}
\usepackage{hyperref}
\tolerance=1000
\providecommand{\alert}[1]{\textbf{#1}}

\title{computation}
\author{Brian Rowe}
\date{\today}
\hypersetup{
  pdfkeywords={},
  pdfsubject={},
  pdfcreator={Emacs Org-mode version 7.8.06}}

\begin{document}

\maketitle

\setcounter{tocdepth}{3}
\tableofcontents
\vspace*{1cm}
\section{Overview}
\label{sec-1}
\subsection{Images}
\label{sec-1-1}


A computer image consists of pixels. Each pixel has three color
components: red, green, and blue. Our calculation operates on a single
color at a time.  When the calculation is complete we use the Java
language's \href{http://docs.oracle.com/javase/6/docs/api/java/awt/image/BufferedImage.html}{BufferedImage} to create a \href{http://www.libpng.org/pub/png/}{PNG image}.
\subsection{Our Data}
\label{sec-1-2}


Our data consists of N distinct data sets. Each of these sets, $i$,
describes the geographic density of various alleles. In a nutshell,
the values, $d_i(x,y)$, in each data set are positively correlated to
the population density of a particular tribe of chimpanzee.
\subsection{User Input}
\label{sec-1-3}
\subsubsection{Ascii File}
\label{sec-1-3-1}
\subsubsection{Color Codes}
\label{sec-1-3-2}


In order to visualize the geographic territory of each tribe, we
assign to each a particular color, $C_i=(r_i,g_i,b_i)$.
\section{Computation}
\label{sec-2}
\subsection{Normalized Data Vector}
\label{sec-2-1}


For each pixel we normalize the associated data:



$Q_s^i(x,y)$

\end{document}
